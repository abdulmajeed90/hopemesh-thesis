\chapter{Introduction}
\section{Personal motivation}
This thesis describes the analysis, enhanced design and implementation of an existing microcontroller based mesh solution \cite{korniowski}. The current solution showed.

\section{Research overview}

\chapter{Evaluation}
\section{Existing solution}
\section{Assumptions}
\section{Requirements}

\chapter{Hardware Design}
\section{RAM}
\begin{itemize}
\item Harvard architecture
\item RAM bus
\item Latch
\end{itemize}
\section{USB Serial Device}
\section{RFM12B Radio}
\section{Keyboard}

\chapter{Software Modules}
\section{UART}
\section{SPI}
\section{RFM12 Driver}
\section{Watchdog}
\section{Clock}
\section{Shell}

\chapter{Software Algorithms}
\section{Protothreads}
Concurrently executing tasks \\
Problem: No operating system \\
"Traditional" embedded implementations are using state machines. Especially the existing thesis uses state machine based algorithms a lot, although the author does not mention this fact at all. \\

Alternatives: \\
- Heavyweight: Real Operating System. Enumarate them and compare ... \\
- Lightweight: Thread implementations. Problem: Each thread has its own stack which consumes a lot of memory. \\
- More Lightweight: Protothreads. Best compromise between classical state machines and real threads. \\

\section{Ring Buffers}
\section{Half-Duplex Radio Access (Petri Net)}

\chapter{Network Stack}
\section{Layer 2a: MAC Layer}
\section{Layer 2b: Logical Link Control}
\section{Layer 3: Batman Routing}
\section{Layer 7: Application}

\chapter{Research}
\section{Simulations}
\subsection{Shell}
\subsection{Routing}
\subsection{Radio Transmission}
\section{Mesh evaluation}
\section{Results}

\chapter{Conclusion}

