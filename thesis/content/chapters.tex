\chapter{Introduction}
\section{Personal motivation}
This thesis describes the analysis, enhanced design and implementation of an existing microcontroller based mesh solution \cite{korniowski}. The current solution showed.

\section{Research overview}

\chapter{Evaluation}
\section{Existing solution}
\section{Assumptions}
\section{Requirements}

\chapter{Hardware Design}
\section{RAM}
\begin{itemize}
\item Harvard architecture
\item RAM bus
\item Latch
\end{itemize}
\section{USB Serial Device}
\section{RFM12B Radio}
\section{Keyboard}

\chapter{Software Modules}
\section{UART}
\section{SPI}
\section{Watchdog}
\section{Clock}
\section{Shell}
\section{Network Stack}
\section{RFM12 Driver}

\chapter{Software Algorithms}
\section{Protothreads}
Designing a software system that executes on embedded micro-controllers implies a lot of challenges when many software modules are involved and complexity grows. The conceptually defined modules must be somehow implemented. If the micro-controller lacks an operating system then there is no possibility of using provided abstractions and APIs for module orchestration and execution. Another challenge are limited hardware resources which prevent the deployment of many existing operating system kernels. Basically there are two types of execution models which can be implemented in micro-controllers:

\minisec{Sequential execution}

\begin{figure}
    \centering
    \import{images/}{sequential_execution.pdf_tex}
    \caption{Sequential execution model}
\end{figure}

This type sequentially executes all modules starting from the first module until the last one. Once the last module ends the execution starts again from the first module. The pseudo code for this execution model is visible in listing \ref{lst:sequential execution}.

\noindent\begin{minipage}{\textwidth}
\begin{lstlisting}[frame=single,caption=Sequential execution pseudo code,label=lst:sequential execution]
function main() {
    while(true) {
        module_1();
        module_2();
        ...
        module_n();
    }
}
\end{lstlisting}
\end{minipage}

\minisec{Concurrent execution}

\begin{figure}
\centering
\import{images/}{concurrent_execution.pdf_tex}
\caption[]{Concurrent execution model}
\end{figure}

"Traditional" embedded implementations are using state machines. Especially the existing thesis uses state machine based algorithms a lot, although the author does not mention this fact at all. \\

Alternatives: \\
- Heavyweight: Real Operating System. Enumerate them and compare ... \\
- Lightweight: Thread implementations. Problem: Each thread has its own stack which consumes a lot of memory. \\
- More Lightweight: Protothreads. Best compromise between classical state machines and real threads. \\

\section{Ring Buffers}
\section{Half-Duplex Radio Access (Petri Net)}

\chapter{Network Stack}
\section{Layer 2a: MAC Layer}
\section{Layer 2b: Logical Link Control}
\section{Layer 3: Batman Routing}
\section{Layer 7: Application}

\chapter{Research}
\section{Simulations}
\subsection{Shell}
\subsection{Routing}
\subsection{Radio Transmission}
\section{Mesh evaluation}
\section{Results}

\chapter{Conclusion}
