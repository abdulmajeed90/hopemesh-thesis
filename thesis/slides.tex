\begin{frame}
\titlepage
\end{frame}

\begin{frame}{Outline}
\tableofcontents
\end{frame}

\section{Introduction}
\begin{frame}{Goals}

\begin{exampleblock}{Main Goal}
Research and implement a fail-safe wireless mesh network prototype using embedded technologies.
\end{exampleblock}

\begin{itemize}
    \item \textbf{Hardware:} Research and develop easily reproducible hardware design.
    \item \textbf{Software:} Research, analyze and implement enhanced concurrent algorithms.
    \item \textbf{Network:} Implement a pragmatic network stack and a B.A.T.M.A.N. based routing algorithm.
    \item \textbf{Simulation:} Make it possible to simulate algorithmic behaviour on x86 based PCs.
\end{itemize} 
\end{frame}

\begin{frame}{Project mind map}
\begin{center}
\scalebox{0.56} {
    \begin{tikzpicture}[auto,swap]
    \path[mindmap, concept color=lightgray!80]
node[concept] {HopeMesh}
[clockwise from=45]
child[concept color=lightgray!70] {node[concept] {\textbf{Hardware}}
    [clockwise from=180]
    child[concept color=lightgray!60] {node[concept] {CPU}}
    child[concept color=lightgray!60] {node[concept] {Radio}}
    child[concept color=lightgray!60] {node[concept] {RAM}}
    child[concept color=lightgray!60] {node[concept] {...}}
}
child[concept color=lightgray!70] {node[concept] {Software}
    child {node[concept] {\textbf{Modules}}
        child {node[concept] {UART}}
        child {node[concept] {SPI}}
        child {node[concept] {...}}
    }
    child {node[concept] {\textbf{Core algorithms}}
        [clockwise from=10]
        child {node[concept] {Concurrency}}
        child {node[concept] {Ring buffers}}
        child {node[concept] {Radio Access}}
    }
    child {node[concept] {\textbf{Network}}
        [clockwise from=330]
        child {node[concept] {MAC}}
        child {node[concept] {LLC}}
        child {node[concept] {...}}
    }
    child {node[concept] {\textbf{Simulations}}
        [clockwise from=240]
        child {node[concept] {Shell}}
        child {node[concept] {Routing}}
        child {node[concept] {...}}
    }
}
;

    \end{tikzpicture}
}
\end{center}
\end{frame}

\section{Hardware}
\begin{frame}{Hardware modules}
\begin{itemize}
    \item \textbf{CPU:} ATMega162: Includes the XMEM extension allowing to use external RAM natively.
    \item \textbf{RAM:} 32kB SRAM: Connected to the CPU using a Latch.
    \item \textbf{Periphery:} USB connection to PC based terminal emulators, LCD and PS/2 keyboard connection.
    \item \textbf{Wireless connection:} RFM12B radio module from HOPERF.
\end{itemize} 
\end{frame}

\begin{frame}{PCB - Printed Circuit Board}
\begin{center}
\scalebox{0.59} {
    \import{figures/}{2nd_rev_pcb.pdf_tex}
}
\end{center}
\end{frame}

\section{Software}
\begin{frame}{Module Architecture}
\begin{center}
\scalebox{0.8} {
    \import{figures/}{modules_component_diagram.pdf_tex}
}
\end{center} 
\end{frame}

\begin{frame}{Concurrency Model}
\begin{block}{Sequential Execution}
Executes modules inside an infinite main loop starting from the first module until the last one.
Once the last module ends the execution starts again from the first module.
\end{block}
\begin{block}{Concurrent Execution}
The main function only initializes and launches concurrent modules. Modules run in their own stack space and can execute individual main loops.
\end{block}
\end{frame}

\begin{frame}{Petri net driver}
\begin{center}
\scalebox{0.8} {
    \import{figures/}{rfm12_petri.pdf_tex}
}
\end{center}
\end{frame}

\section{Network}

